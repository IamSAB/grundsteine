\chapter{Auf den Felsen gegründet}

\begin{figure}[h]
  \includegraphics[width=\textwidth]{auf-den-felsen-gegruendet/teaser.jpg}
  \centering
\end{figure}

Wir haben am eigenen Leib erlebt, dass man in einer bestimmten Situation durch nichts Gottes Kraft besser freisetzen kann, als durch eine Proklamation Seines Wortes im Glauben. 
In der Tat habe ich schon oft gesagt, und werde auch nicht müde es immer wieder zu sagen, dass ich ernsthaft bezweifele, ob Ruth oder ich heute noch am Leben wären, wenn wir es nicht gelernt hätten, diese Waffe der Proklamation einzusetzen.

Wir beide haben zu verschiedenen Zeiten eine schwere Krankheit durchgemacht.
Ich zog mir etwas zu – besser gesagt, wurde mir etwas angehängt, das in der Regel zum Tod führt.
Ich habe es den Ärzten, Gott und der Kraft der Heiligen Schrift zu verdanken, dass ich heute hier gesund vor euch stehe und dem Herrn diene.

So, heute Vormittag wollen wir Jesaja 55,10 - 11 proklamieren, eine Passage, die uns für diese Lehrserie ganz besonders passend erscheint.

\begin{quotation}
  Denn gleichwie der Regen und der Schnee vom Himmel fällt und nicht wieder dahin zurückkehrt, bis er die Erde getränkt und befruchtet und zum Grünen gebracht hat und dem Sämann Samen gegeben hat und Brot dem, der isst – genauso soll auch mein Wort sein, das aus meinem Mund hervorgeht: es wird nicht leer zu mir zurückkehren, sondern es wird ausrichten, was mir gefällt, und durchführen, wozu ich es gesandt habe!
  \sourceatright{Jes 55,10-11}
\end{quotation}

Der Titel dieser ersten Lehreinheit lautet, „Auf den Felsen Gegründet“. Dieses Thema stellen wir an den Anfang. Die Bibel ist ein Vorbild was gute Lehre anbetrifft, und sie hält sich an verschiedene Lehrprinzipien. Ein solches Prinzip besagt, dass man mit dem Bekannten beginnt und die Zuhörer dann zum Unbekannten weiterführt. Man beginnt nie mit dem Unbekannten, sondern mit Dingen die bekannt sind, und von dort aus, geht man dann weiter zum Unbekannten.

Die Bibel realisiert dieses Prinzip indem sie ganz einfache, allgemein bekannte, alltägliche Erfahrungen und Dinge aufgreift und ihnen eine geistliche Bedeutung verleiht. 
Dafür gibt es verschiedene Beispiele: Die Bibel spricht von einem Bauern, der sät, von einem Fischer, der mit einem Wurfnetz Fische fängt, oder von einem Soldaten, der seine Rüstung anzieht. 
In einem völlig anderen Kontext spricht sie beispielsweise von einer Braut, die sich auf die Hochzeit vorbereitet. 
Das sind nur einige Beispiele, die dieses Prinzip veranschaulichen.

Das konkrete Beispiel, das ich jetzt herausgreifen möchte ist der Bau eines Hauses. 
Ich denke, dass dieses Bild für das Christenleben in der Bibel mindestens genauso oft wie alle anderen Bilder auftaucht. 
Schlagen wir deshalb als erstes den Judasbrief auf.
Es handelt sich hier um ein Wort der Ermahnung an uns Christen.

\begin{quotation}
  Ihr aber, Geliebte, erbaut euch auf euren allerheiligsten Glauben und betet im Heiligen Geist; bewahrt euch selbst in der Liebe Gottes … 
  \sourceatright{Jud 20-21}
\end{quotation}

Die Bibel sagt also hier, wir müssten uns auf unserem heiligsten Glauben aufbauen.
Das ist eine Anwendungsmöglichkeit dieses Bildes vom Hausbauen. 
Wir haben die Verantwortung uns selbst aufzubauen.

In Epheser ist von einem heiligen Tempel im Herrn die Rede.

\begin{quotation}
  ... [in Christus] zusammengefügt, wächst [der ganze Bau] zu einem heiligen Tempel im Herrn, in Ihm werdet auch ihr miterbaut zu einer Wohnung Gottes im Geist. 
  \sourceatright{Eph 2,21-22}
\end{quotation}

Hier heißt es, die Gemeinschaft aller Christen solle im Heiligen Geist zu einem Tempel aufgebaut werden, an dem Gott wohnen kann.

Dann spricht Petrus in ersten Petrusbrief schließlich noch von Jesus als lebendigen Stein und sagt:

\begin{quotation}
  Da ihr zu ihm gekommen seid, zu dem lebendigen Stein, der von den Menschen zwar verworfen, bei Gott aber auserwählt und kostbar ist, so lasst auch ihr euch nun als lebendige Steine aufbauen, als ein geistliches Haus, als ein heiliges Priestertum, um geistliche Opfer darzubringen, die Gott wohlgefällig sind durch Jesus Christus. 
  \sourceatright{1. Petr 2,4-5}
\end{quotation}

Hier wird jeder einzelne von uns mit einem lebendigen Stein verglichen, die alle zusammen zu einem heiligen Tempel zusammengebaut werden, in dem der Herr wohnen wird.

Und dann ein letztes Beispiel finden wir in Apostelgeschichte Kapitel 20. 
Die Abschiedsrede des Paulus an die Ältesten von Ephesus, für die er eine ganz besondere Liebe hatte, weil sich nämlich sein Dienst in Ephesus am nachhaltigsten ausgewirkt hatte. 
Mit dieser Rede in Apostelgeschichte 20 nimmt er Abschied und sagt den Ältesten, er werde sie in diesem Leben nie wieder sehen. 
Das war ein sehr bewegender Augenblick für alle beteiligten. 
In Vers 32 lesen wir seine letzten Worte, die er ihnen noch hinterlassen wollte.

\begin{quotation}
  Und nun, Brüder, übergebe ich euch Gott und dem Wort seiner Gnade, das die Kraft hat, euch aufzuerbauen und ein Erbteil zu geben unter allen Geheiligten.
  \sourceatright{Apg 20,32}
\end{quotation}

Paulus sagt hier, der Aufbau geschehe vor allem durch das Wort der Gnade Gottes, also durch die Bibel. 
Er sagt weiterhin, es könne uns aufbauen und uns ein Erbteil geben unter all jenen, die durch den Glauben an Jesus Christus für ihn abgesondert sind.

Gut, ich bin zwar kein Baumeister, aber eins weiß ich: Bei jedem dauerhaften Bauwerk, sei es aus Ziegel oder Stein oder Beton oder Holz ist das Fundament von entscheidender Bedeutung. 
Die Bibel geht ganz speziell auf dieses Fundament ein. 
Die Frage nach dem richtigen Fundament ist für jeden von uns ungemein wichtig, denn das Fundament entscheidet, wie groß und schwer das Gebäude sein darf, das darauf aufgebaut wird. Das Fundament macht die Vorgaben für den restlichen Bau. 
Dasselbe gilt für das Leben eines Christen. 
Du kannst nur in dem Maße ein erfolgreiches Leben als Christ bauen, wie es dein Fundament gestattet. 
Das ist der springende Punkt: Wie sieht dein Fundament aus? Hast du das richtige Fundament gelegt?

\section{Das Fundament richtig legen}

Nun gibt es nur ein angemessenes und ausreichendes Fundament. 
Es leistet allen Kriterien genüge. 
Es ist eine Person. 
Diese Person ist Jesus Christus. 
Im 1. Korintherbrief schreibt Paulus in Kapitel 3: Er verwendet zwei Metaphern.
Zunächst eine aus der Landwirtschaft, und anschließend eine aus dem Bauwesen. 
Er sagt in Vers 9: „Denn wir sind Gottes Mitarbeiter;“ – wir arbeiten mit Gott zusammen – „ihr aber seid Gottes Ackerfeld“ – das ist die Metapher aus der Landwirtschaft. 
Dann geht er weiter – „und [ihr seid] Gottes Bau“ – das ist die Metapher aus dem Bauwesen, die er folgendermaßen weiter führt: „Gemäß der Gnade Gottes, die mir gegeben ist, habe ich als ein weiser Baumeister den Grund gelegt;“ – im griechischen steht hier das Wort „Architekt“ – „ein anderer aber baut darauf.
Jeder aber gebe Acht, wie er darauf aufbaut. 
Denn einen anderen Grund kann niemand legen außer dem, der gelegt ist, welcher ist Jesus Christus.“

Paulus sagt demnach, es gebe nur eine Grundlage, ein Fundament für das Christenleben, nämlich Jesus selbst. 
Alles was nicht auf dieses Fundament gebaut wird, wird nicht standhalten, wenn die Zeit vergeht und Bewährungsproben kommen. 
Deshalb ist es äußerst wichtig, dass jeder von uns nachprüft, worauf er sein Leben aufgebaut hat. 
Haben wir wirklich auf den Herrn Jesus Christus aufgebaut? 
Haben wir eine persönliche Beziehung zu Jesus, und eine persönliche Erkenntnis von Jesus, die uns in die Lage versetzt ganz persönlich mit ihm zu kommunizieren?

Jesus als Fundament – dieses Thema ist ausgesprochen wichtig. 
Deshalb möchte ich mich nun einige Zeit mit der Frage beschäftigen, wie können wir dieses Fundament bekommen? 
Jesus als Fundament unseres Lebens.

Ich möchte dich auffordern, dein Leben, deinen geistlichen Zustand und deinen geistlichen Erfahrungshorizont unter die Lupe zu nehmen und nachzuprüfen, ob deine Beziehung zum Fundament wirklich stimmt.

Anhand von Matthäus 16 möchte ich euch nun einige grundlegende Dinge nahebringen. 
In Vers 13 und folgende redet Jesus mit seinen Jüngern, und es heißt:

\begin{quotation}
  Als aber Jesus in die Gegend von Cäsarea Philippi gekommen war, fragte er seine Jünger und sprach: Für wen halten die Leute mich, den Sohn des Menschen? 
  Sie sprachen: Etliche für Johannes den Täufer; andere aber für Elia; noch andere für Jeremia oder einen der Propheten.
  \sourceatright{Matt 16,13-14}
\end{quotation}

Dann wird er sehr persönlich:

\begin{quotation}
  Da spricht er zu ihnen: Ihr aber, für wen haltet ihr mich? 
  Da antwortete Simon Petrus und sprach: Du bist der Christus, der Sohn des lebendigen Gottes!
  \sourceatright{Matt 16,15-16}
\end{quotation}

Das war ein entscheidender Augenblick im Leben des Petrus und in der Geschichte der Christenheit.

\begin{quotation}
  Du bist der Christus, [der Messias] der Sohn des lebendigen Gottes! 
  Und Jesus antwortete und sprach zu ihm: Glückselig bist du, Simon, Sohn des Jona; denn Fleisch und Blut hat dir das nicht geoffenbart, sondern mein Vater im Himmel!
  Und ich sage dir auch: Du bist Petrus, und auf diesen Felsen will ich meine Gemeinde bauen, und die Pforten des Totenreiches sollen sie nicht überwältigen.
  \sourceatright{Matt 16,16b-18}
\end{quotation}

Jesus veranschaulicht also anhand dieser Begegnung mit Petrus, wie wir unser Fundament in Jesus Christus selbst legen können. 
Zunächst jedoch einige Kommentare zu den Wörtern, die in dieser Passage auftauchen. 
Jesus sagt in Vers 18, „Du bist Petrus“ – im griechischen heißt das petros – „und auf diesem Felsen“ – griechisch petra – „werde ich meine Gemeinde bauen“.
Viele Menschen meinen, Petrus sei das Fundament der Gemeinde Jesu. Wenn dies stimmt, wäre es meiner Anschauung nach jedoch ein sehr wackeliges Gebäude, denn nur ein wenig später weißt Jesus Petrus zu Recht, und sagt zu ihm, „gehe hinter mich Satan“.
Noch später verleugnet Petrus den Herrn drei Mal, und nach der Auferstehung musste Paulus Petrus sogar ermahnen, aus Angst um seine jüdischen Mitbrüder keine Kompromisse bezüglich der Wahrheit des Evangeliums einzugehen.

\section{Der ewige Felsen}

Ich bin wirklich sehr dankbar, dass der Leib Christi nicht auf Petrus oder einem anderen Menschen aufgebaut ist. 
Was diese Stelle eigentlich aussagt wird im Griechischen, also im uns zur Verfügung stehenden Originaltext, sehr deutlich: „Du bist Petrus“ – petros – „und auf diesem Felsen“ – petra – „werde ich meine Gemeinde bauen.“
Das griechische Wort petros bezeichnet einen Stein, höchstens noch einen Felsbrocken aber nichts Größeres.
In der Regel wäre das ein Stein, wie man ihn früher aufhob, um jemanden zu steinigen.
Petra bezieht sich jedoch auf eine zerklüftete Felsspitze, die von der Felssohle aufragt; meist eine Klippe oder etwas in der Größenordnung.
Das entscheidende dabei ist jedoch, dass es ein Teil der Felssohle ist. Was ist die Felssohle?
Genau das, was Petrus soeben begriffen hatte, nämlich die Erkenntnis Jesu als der, der er ist, so wie ihn uns nur der Heilige Geist offenbaren kann.
Niemand kann Jesus als den erkennen, der er wirklich ist, wenn Gott Vater es ihm nicht durch den Heiligen Geist offenbart.

Unser christlicher Glaube muss also auf diesen petra, auf diese Felssohle gebaut sein.
Es geht hierbei um eine persönliche Begegnung mit Jesus und eine persönliche Offenbarung von ihm, nicht als Sohn des Zimmermanns, nicht als historische Persönlichkeit, sondern als ewiger nicht-geschaffener Sohn Gottes.
Wenn wir auf diesen Felsen bauen wollen, müssen wir zunächst zu dieser Erkenntnis gelangen und dieselbe Erfahrung machen, die Petrus machte.
Ich habe es schon oft gesagt: Du kannst einer Gemeinde beitreten, du kannst eine religiöse Zeremonie durchlaufen, du kannst ein Gebet sprechen, und trotzdem bleibst du der Alte.
Doch wenn du Jesus begegnest wirst du verändert werden.
Niemand kann Jesus begegnen und dabei der Alte bleiben.
Deshalb muss sich jeder von uns fragen: Hatte ich schon einmal diese lebensverändernde, persönliche Begegnung mit dem Herr Jesus Christus?
Meinem Verständnis nach durchlief Petrus im Verlauf dieser Begegnung vier aufeinander folgende Phasen.

\section{Die persönliche Konfrontation}

Erstens, die Konfrontation. Jesus und Petrus begegneten einander von Angesicht zu Angesicht.
Es gab keinen Vermittler zwischen ihnen, keinen Priester; niemand stand zwischen ihnen.
Petrus war direkt und persönlich mit Jesus konfrontiert. Genau an diesen Punkt müssen wir auch kommen.
Jesus sagt an anderer Stelle: „Ich bin die Tür. Wenn jemand durch mich eingeht, so wird er errettet werden.“
Es gibt nur einen Weg in das Reich Gottes: Durch die Tür.
Diese Tür ist keine Gemeinde, keine Lehre sondern Jesus.
Er sagt, „Ich bin die Tür“.

\section{Die persönliche Offenbarung}

Auf die Konfrontation folgt eine Offenbarung, eine Offenbarung die Petrus von Gott Vater durch den Heiligen Geist geschenkt wurde.
Jesus sagte, „Fleisch und Blut haben dir das nicht offenbart“.
Du kannst nicht durch deinen gesunden Menschenverstand zu dieser Erkenntnis gelangen.
Es ist eine Offenbarung nötig.
Dies ist ein wesentlicher Punkt: Ohne persönliche Offenbarung kann niemand Jesus in seiner wahren Identität als ewiger Sohn Gottes erkennen.
Du kannst Theologie studieren, eine Bibelschule besuchen oder gar in den geistlichen Dienst gehen, doch ohne diese persönliche Offenbarung Jesu kannst du ihn unmöglich kennen und erkennen.
Diese Offenbarung kommt von Gott Vater durch den Heiligen Geist.

Ich möchte dich fragen, du brauchst jetzt nicht zu antworten: Hattest du diese persönliche Begegnung mit Jesus?
Ich hatte sie.
Vor mehr als 50 Jahren im Zweiten Weltkrieg begegnete ich Jesus mitten in der Nacht in einer Armeekaserne.
Ich hatte keine Ahnung von christlicher Lehre; der christliche Jargon war mir fremd; ich konnte nicht sagen, ich sei bekehrt oder wiedergeboren.
All das lernte ich erst später.
Aber ich sage euch eins: Ich wurde verändert, radikal und anhaltend. Ich wurde nicht vollkommen.
Ich muss euch gestehen, dass ich immer noch nicht perfekt bin, aber ich wurde verändert, und zwar positiv.

\section{Anerkennung der Offenbarung}

Anschließend folgt die Anerkennung dessen, was uns der Heilige Geist aufzeigt.
Wir müssen sagen: „Ja, ich glaube. Ich empfange.“ Wir müssen irgendwie reagieren.
Das geschieht nicht automatisch, es muss etwas in uns geschehen.

\section{Das öffentliche Bekenntnis}

Viertens folgt dann das öffentliche Bekenntnis unseres Glaubens an Jesus. 
Jesus brachte Petrus so weit, dass er sagte, „du bist der Christus, der Messias“. 
Er bekannte seinen Glauben öffentlich. 
Es ist oft die Rede von Leuten, die im Geheimen Christen sind, und ich akzeptiere es auch, dass Leute im Geheimen Christ sind, vor allen in Ländern, wo man mit dem Tod rechnen muss, wenn man sich als Christ zu erkennen gibt.
Ich glaube jedoch nicht, dass jemand auf Dauer im Geheimen Christ sein kann.

Ich lese euch vor, was Jesus in Matthäus 10 sagte:

\begin{quotation}
  Jeder nun, der sich zu mir bekennt vor den Menschen, zu dem werde auch ich mich bekennen vor meinem Vater im Himmel; wer mich aber verleugnet vor den Menschen, den werde auch ich verleugnen vor meinem Vater im Himmel.
  \sourceatright{Matthäus 10,32-33}
\end{quotation}

Es ist typisch Jesus, dass er uns nicht drei sondern nur zwei Alternativen lässt: Entweder bekennen oder verleugnen. 
Wenn du ihn in einer passenden Situation nicht bekennst, verleugnest du ihn im Grunde. 
Jeder von uns muss früher oder später an den Punkt kommen, dass er seinen Glauben an den Herrn Jesus Christus offen bekennt. 
Für viele ist das ein kritischer Moment.

Nachdem ich in der Armee Christ geworden war stellte ich fest, dass es immer das Beste war jedermann gleich bei der ersten Begegnung wissen zu lassen, dass man Christ ist, dann braucht man nämlich nie nachher zu ihm hingehen und sagen, „weißt du, ich habe es dir neulich nicht gesagt, aber…“

Wo ich auch war kniete ich jeden Abend auf der Stube neben dem Bett nieder und betete, ohne dass dies ein religiöser Akt gewesen wäre. 
So wussten alle anderen sofort, mit wem sie es zu tun hatten. Das war viel leichter als anders herum. 
Ich sah andere Christen, die belanglose Dinge redeten, und nicht frisch heraus sagten, dass sie gläubig wären. 
Für sie war es weitaus schwieriger nachher noch einmal hinzugehen und ihren Glauben zu bekennen. 
Ich möchte euch diese Vorgehensweise ans Herz legen. 
Du brauchst dich nicht an die Straßenecke zu stellen und zu predigen – du brauchst kein Prediger zu werden – sei einfach wer du bist, ob Hausfrau oder Student. 
Doch lass die Leute wo immer du bist wissen, dass du an Jesus als Sohn Gottes glaubst.

Ich wiederhole nun kurz die vier aufeinander folgenden Phasen dieser Begegnung, die so grundlegend ist, weil wir nämlich auf diese Weise Jesus als Fundament unseres persönlichen Lebens legen.

\begin{enumerate}
  \item Am Anfang stand die Konfrontation.
  \item kam die Offenbarung, die Gott Vater durch den Heiligen Geist schenkt.
  \item reagierte Petrus mit einer Anerkennung.
  U\item folgte sein öffentliches Bekenntnis.
\end{enumerate}

Ihr fragt nun vielleicht, ist so eine Offenbarung auch heute noch möglich?
Können Menschen wie du und ich Jesus so real und persönlich kennen lernen wie Petrus und die anderen Jünger?
Hier sind zwei wichtige Dinge zu beachten: Zuerst einmal wurde Jesus Petrus nicht als Sohn des Zimmermanns offenbart.
Er hatte ihn als solchen schon eine Zeit lang gekannt.
Er wurde ihm als der ewige Sohn Gottes offenbart.
Die Bibel sagt:
\begin{quotation}
  Jesus Christus ist derselbe gestern und heute und in Ewigkeit!
  \sourceatright{Hebräer 13,8}
\end{quotation}
Er hat sich nie geändert, und wird sich auch nie ändern.
Es geht also nicht um Sprache oder Kultur oder Kleidung sondern um die ewige Person Jesu.
Das hat Petrus hier vielleicht zum ersten Mal in seinem Leben erkannt, er bekam tatsächlich eine Offenbarung darüber wer Jesus ist.
Zweitens, er bekam die Offenbarung durch den Heiligen Geist.
Die Bibel bezeichnet den Heiligen Geist als den ewigen Geist, den zeitlosen Geist – Zeit, Mode, Geschichte, Bräuche, Sprache haben keinerlei Auswirkungen auf den Heiligen Geist. 

Aus diesen beiden Gründen können du und ich genau dieselbe, direkte, persönliche Offenbarung über Jesus bekommen wie sie damals Petrus bekam, erstens weil der ewige Sohn Gottes offenbart wird, zweitens weil es der ewige Geist ist, der ihn offenbart.

\section{Auf das Fundament bauen}

Nun kommen wir zum nächsten, wichtigen, praktischen Punkt: Du hast das Fundament gelegt, doch wie baust du nun darauf? 
Ihr erinnert euch, dass es in den Metaphern, die wir eingangs erwähnten immer ums Bauen ging. 
Das ist also die nächste, absolut entscheidende, praktische Frage: Wie baut man auf das Fundament?

Ich möchte dazu ein Gleichnis Jesu aus Matthäus 7 zu Rate ziehen.
Das bekannte Gleichnis vom weisen und vom törichten Mann, die beide ein Haus bauten jedoch sehr unterschiedlich dabei vorgingen.

\begin{quotation}
  Ein jeder nun, der diese meine Worte hört und sie tut, den will ich mit einem klugen Mann vergleichen, der sein Haus auf den Felsen baute.
  [Auf die Felssohle, petra.]
  Als nun der Platzregen fiel und die Wasserströme kamen und die Winde stürmten und an dieses Haus stießen, fiel es nicht; denn es war auf den Felsen gegründet.
  Und jeder, der diese meine Worte hört und sie nicht tut, wird einem törichten Mann gleich sein, der sein Haus auf den Sand baute.
  Als nun der Platzregen fiel und die Wasserströme kamen und die Winde stürmten und an dieses Haus stießen, da stürzte es ein, und sein Einsturz war gewaltig.
  \sourceatright{Matthäus 7,24-27}
\end{quotation}

Zunächst einmal ist es wichtig hervorzuheben, dass beide Häuser dieselbe Bewährungsprobe bestehen mussten.
Keinem der beiden Häuser wurde dieser Härtetest erspart.
Derselbe Sturm rüttelte an beiden Häusern, und ich sage es euch, das Christenleben hat seine Stürme.
Ihr werdet Stürme erleben.
Gott hat nie versprochen, dass euch das erspart bleiben würde.
Ja, Paulus und Barnabas sagten in der Apostelgeschichte zu einer Gemeinde: „Wir müssten durch viele Trübsale in das Reich Gottes eingehen.“
Wenn du auf deinem Weg keine Trübsale durchmachst, ist es fraglich, ob er überhaupt auf das Reich Gottes zuführt, denn genau das sagte Paulus ja, wir müssen durch viele Trübsale in das Reich Gottes eingehen.
Es wäre hier fehl am Platze, zu erläutern, warum wir Trübsale durchmachen, doch glaubt mir, Gott bezweckt etwas damit.
Wenn du dich derzeit in einer solchen Phase befindest, dann gib nicht auf.
Gott wird dich durch tragen, und am Ende wirst du feststellen, dass er an dir gewirkt und dir Dinge beigebracht hat, die du anderweitig nie gelernt hättest. Woher weiß ich das?
Aus eigener Erfahrung, ich predige nur äußerst selten graue Theorie.

Wie baut nun dieser weise Mann? 
Ganz einfach, indem er die Worte Jesu, die Worte der Bibel, hört und sie tut.
Sei nicht nur Hörer des Wortes, denn die Bibel hat keine Verheißungen für bloße Hörer.
Höre das Wort und tue es.
Es geht um die Praxis, es geht darum, die Lehre der Bibel, die Lehre Jesu, in deinem eigenen Leben anzuwenden. Wenn du dies tust und hierin weiter gehst, wirst du feststellen, dass Gott dir immer wieder neue Bereiche aufschließt, in denen du die Wahrheit praktisch umsetzen musst.
Ich bin seit mehr als 50 Jahren Christ, doch Gott zeigt mir immer wieder neue Möglichkeiten, sein Wort praktisch anzuwenden, und neue Lebensbereiche, in denen ich es anwenden muss.
Mein Haus ist noch nicht fertig, es ist noch im Bau, aber ich danke Gott dafür, dass es schon etliche Stürme heil überstanden hat.

Eine andere Geschichte Jesu aus Lukas 6 ist der eben zitierten sehr ähnlich, bis auf einem wichtigen Zusatz. Wiederum sagt Jesus:

\begin{quotation}
  Was nennt ihr mich aber ‚Herr, Herr‘ und tut nicht, was ich euch sage?
  \sourceatright{Lukas 6,46}
\end{quotation}

Lippenbekenntnis reicht nicht!

Das ist eine wichtige Frage. 
Es ist sinnlos Jesus „Herr“ zu nennen, wenn du ihm nicht gehorchst, denn schon der Titel „Herr“ bezeichnet jemanden, den man gehorchen muss. 
Jesus sagt, hüte dich vor einem Lippenbekenntnis, das auf deinem Lebensstil keinerlei Auswirkungen hat.

Das Gleichnis geht weiter:

\begin{quotation}
  Jeder, der zu mir kommt und meine Worte hört und sie tut – ich will euch zeigen, wem er gleich ist. Er ist einem Menschen gleich, der ein Haus baute und dazu tief grub und den Grund auf den Felsen legte. Als nun eine Überschwemmung entstand, da brandete der Strom gegen dieses Haus, und er konnte es nicht erschüttern, weil es auf den Felsen [die Felssohle, petra] gegründet war. Wer aber hört und nicht tut, der ist einem Menschen gleich, der ein Haus auf das Erdreich baute, ohne den Grund zu legen; und der Strom brandete gegen dasselbe, und es stürzte sofort ein, und der Zusammenbruch dieses Hauses war gewaltig.
  \sourceatright{Lukas 6,47-49}
\end{quotation}

Im Lukastext finden wir ein wichtiges Detail, das im Matthäustext nicht erscheint. Wie vielen von euch ist es aufgefallen?
Hier heißt es, der Mann musste tief graben, um bis zur Felssohle hinunter zu kommen.
Er musste vieles aus dem Weg schaffen, bevor er auf der Felssohle bauen konnte.
Dasselbe gilt vielleicht nicht für alle von uns, aber auf jeden Fall für viele die im Namens-Christentum groß geworden sind.
Wir müssen viele Dinge aus dem Weg räumen bis wir die Felssohle erreichen. 
Andere, die aus einem nicht-christlichen Hintergrund kommen müssen auch Dinge ausräumen, doch die sind anderer Natur.
Meines Erachtens sind es fünf Dinge, die wir beim Graben aus dem Weg räumen müssen:

\subsection{Traditionen}

Als erstes sind es unsere Traditionen.
Nun sind nicht alle Traditionen schlecht, es gibt auch gute Traditionen.
Wir wollen nicht alle Traditionen über Bord werfen, doch Jesus sagte zu seinen Zeitgenossen, „um eurer Traditionen willen habt ihr das Wort Gottes ungültig gemacht. Ihr Glaubt an Traditionen und handelt dem entsprechend, obwohl sie nicht mit der Bibel übereinstimmen.“
Meiner Beobachtung nach würde Jesus heute genau dasselbe zu denselben Juden sagen, „um eurer Traditionen willen, habt ihr das Wort Gottes ungültig gemacht“.
Doch schauen wir nicht nur auf die Juden, denn dasselbe gilt für viele andere Leute, die aus einem christlichen Umfeld stammen.
Wir haben Traditionen geerbt, Verhaltensweisen, Gepflogenheiten, eine Art zu reden, die nicht notwendiger Weise mit der Bibel übereinstimmen.
Wir müssen uns sehr sorgfältig darauf prüfen.
Ich könnte euch nun etliche praktische Anregungen geben, doch ich denke, das spare ich mir.

\subsection{Vorurteile}

Das zweite, das wir ausmerzen müssen, sind Vorurteile.
Es gibt keinen unter uns, der nicht irgendwann in seinem Leben Vorurteile gehabt hätte. Vielleicht bist du sie schon losgeworden, ich weiß es nicht.
Es gibt jedenfalls alle möglichen Vorurteile, zum Beispiel Vorurteile gegenüber bestimmten Bevölkerungsgruppen, wie sie heutzutage auf der Welt leider gang und gäbe sind.
Wir wissen, dass solche Vorurteile in Ländern wie Südafrika, wo es inzwischen fantastische Veränderungen gegeben hat, bestimmte Leute sogar aus dem Leib Christi ausgrenzten.
Ein entsetzlicher Gedanke, doch auch in anderen Ländern gibt es diese Ressentiments.
Vorurteile gegen andere Bevölkerungsgruppen hat es in den USA in Hülle und Fülle gegeben, und gibt es vielfach auch heute noch.

Ich stamme aus Großbritannien, ich kann euch sagen, dass die Briten auch ihre Vorurteile haben.
In meiner Kindheit bekam ich viele davon mit auf den Weg. Ich musste tief graben um sie los zu werden.
Meine eigene Familie ist aufs engste mit Indien verbunden.
Alle meine Vorfahren dienten in der britischen Armee in Indien.
Ich erinnere mich noch, wie ich in meiner Naivität als 12 jähriger Junge einmal am Mittagstisch sagte, „ich verstehe nicht, warum wir keinen Inder zum Mittagessen einladen“. Meine Familie war entsetzt.
Ich fragte mich, warum das so sei? Später erkannte ich dann, dass dies ein Vorurteil ist.
Glaubt mir, ich begegne vielen verschiedenen Leuten aus unterschiedlichen Bevölkerungsgruppen, doch nur die wenigsten haben keinerlei Ressentiments gegenüber anderen Bevölkerungsgruppen.

Dann gibt es die Vorurteile gegenüber anderen Konfessionen.
Die meisten von uns reagieren irgendwie negativ auf bestimmte Denominationen.
Meine erste Frau, die inzwischen beim Herrn ist, war Dänin, sie wurde in der dänisch-lutherischen Kirche groß, und tat dann etwas, was in deren Augen etwas ganz entsetzliches war: Sie ließ sich als erwachsene Gläubige taufen, und war somit eine sogenannte Wiedertäuferin.
Da sie Lehrerin an einer staatlichen, dänischen Schule war, ging ihr Fall bis vors Parlament, das darüber entschied, ob sie Lehrerin bleiben dürfte.
Ich muss sagen, dass sie bis an ihr Lebensende einen Kleinkrieg gegen die Lutheraner führte.
Ich rechtfertige das nicht, sondern denke, dass das ein Schwachpunkt von ihr war.

Ich erlebe es ja an mir selbst, dass ich eine bestimmte Haltung gegenüber einem Menschen entwickle, wenn mir nur zu Ohren kommt, er gehöre dieser oder jener Denomination an obwohl ich ihn persönlich nie kennen gelernt habe.
Ich denke mir, er wird sich so und so verhalten, und in diesem Punkt liegt er falsch, und so weiter.
Meine Erfahrung hat mich gelehrt, einen Menschen soweit irgend möglich erst dann zu beurteilen, wenn man ihn kennen gelernt hat.
Ich bin Leuten begegnet, die zwar aus der falschen Denomination kommen, aber absolut richtig liegen, und Leuten die aus der richtigen Denomination kommen, aber falsch liegen. Deshalb bitte hütet euch vor Vorurteilen gegenüber anderen Konfessionen.

Dann gibt es noch soziale Ressentiments.
Auch hier bin ich selbst wieder ein Beispiel für jemanden, der mit sozialen Ressentiments groß wurde.
Ich war mir dieser Vorurteile gar nicht bewusst.
Ich ging in Großbritannien ans Eaton College und dann an die Universität Cambridge, ich hatte einfach keine Ahnung, wie der Rest der Welt lebte.
Dann steckte man mich in die britische Armee, und ich war mit allen möglichen Leuten zusammen, mit denen ich bislang nie Kontakt gehabt hatte.
Ich erkannte, wie wenig ich im Grunde über meine britischen Landsleute wusste. Ich danke Gott für die Erfahrungen, die ich während dieser fünf ein halb Jahre Militärdienst machen durfte.
Ich wurde dabei etliche soziale Vorurteile los.

Da ich aus einer Familie von Offizieren stamme, war ich an ein bestimmtes Niveau gewöhnt, und immer wenn ich nicht auf diesem Niveau lebte, konnte ich etwas lernen. 
Wenn man sich Menschen ansieht, mit denen man auf ein und demselben Niveau steht, sehen sie alle gleich aus. 
Doch, wenn man sie aus einem anderen Niveau aus betrachtet, sehen sie anders aus. 
Seither versuche ich zu sagen, „Gott, wie sehe ich für die Menschen aus, die mich von einem anderen Niveau aus sehen?“

Es gibt also die unterschiedlichsten Vorurteile wie zum Beispiel auch persönliche Vorurteile. 
Der eine mag keine Leute, die eine laute Stimme haben. 
Der andere mag keine Leute mit roten Haaren. Die meisten von uns haben alle möglichen, dummen, persönlichen Vorurteile. 
Ich habe ein Vorurteil gegenüber Leuten, die geräuschvoll Äpfel kauen. 
Ich kämpfe dagegen an, aber es bleibt latent vorhanden, weil ich dieses Geräusch einfach nicht mag.

\subsection{Vorgefasste Meinungen}

Darüber hinaus gibt es Leute, die vorgefasste Meinungen haben, wie zum Beispiel ein völlig falsches Bild von Jesus. 
Sie sehen in ihm den sanften, lieben, lamm-frommen Jesus, der an Weihnachten Geschenke verteilt. 
Das ist nicht der echte Jesus, er war ganz anders; er schockierte und neigt nach wie vor dazu, unsere Vorurteile und vorgefassten Meinungen aus den Angeln zu heben.

Es gibt vorgefasste Meinungen über viele andere Dinge, zum Beispiel eine vorgefasste Meinung darüber, was es heißt, Christ zu sein. 
Ich beschrieb eben, in welchem Umfeld ich groß wurde. 
Vor diesem Hintergrund kam ich zu dem Schluss, ich würde mir ein lebenslanges Elend aufhalsen, wenn ich Christ würde. 
Ich dachte wie Pat Boone, „der Himmel ist keine siebzig Jahre Elend auf Erden wert“, deshalb schloss ich die Möglichkeit, Christ zu werden, kategorisch aus, bis ich Jesus kennen lernte.

\subsection{Unglaube}

Unglaube ist ein weiterer, sehr gefährlicher Hinderungsgrund. 
Manchmal stelle ich mich gleich am Anfang einer Predigt gemeinsam mit den Zuhörern gegen den Unglauben, weil viele von uns noch in den unterschiedlichsten Bereichen damit zu kämpfen haben. 
Unser Sinn ist nicht wirklich offen für den Glauben.


\subsection{Rebellion}

Der letzte, und meines Erachtens wichtigster Punkt ist Rebellion. Du sagst vielleicht, „aber Bruder Prince, ich bin doch kein Rebell“. 
Oh doch, du bist einer, und so lange du dir dessen nicht bewusst bist, bleibst du einer. 
Ich will diesen Punkt nicht theologisch beleuchten, doch jeder Nachkomme Adams hat von Geburt an einen Rebellen in sich. 
Wir müssen diesen Rebellen erkennen, und gegen ihn vorgehen. 
Gott hat nur ein Mittel gegen diesen Rebellen, wisst ihr welches? 
ie Hinrichtung, genau. 
Er schickt ihn nicht in eine Gemeinde, er lehrt ihn nicht, „das was du nicht willst, das man dir tut, das füge auch keinem anderen zu“, er lässt ihn keine Bibelstelle auswendig lernen, er bringt ihn um. 
Aber es ist die Gnade Gottes, dass diese Hinrichtung bereits vor 2000 Jahren stattfand, als Jesus am Kreuz starb. 
Unser alter Mensch wurde damals mit ihm gekreuzigt. 
Wir müssen an den Punkt kommen, dass wir diesen Rebellen in uns erkennen und bereitwillig hinrichten lassen.

\section{Die Bibel – Gottes Wort}

Nun komme ich auf den Stellenwert der Bibel zu sprechen, weil das mit der wichtigste Punkt im Leben eines Christen ist. 
Wie stehst du zur Bibel? 
Siehst du sie so, wie Jesus sie sah? 
Ich möchte nur kurz Johannes 10 zitieren, ohne den Kontext zu betrachten.
Jesus sagte hier:

\begin{quotation}
  Wenn er [damit ist Gott gemeint] jene Götter nannte, an die das Wort Gottes erging – und die Schrift kann nicht aufgelöst werden.
  \sourceatright{Joh 10,35}
\end{quotation}

Das ist ein sehr wichtiger Vers, da Jesus darin die beiden wichtigsten Bezeichnungen für die Bibel verwendet: „Das Wort Gottes“ und „die Schrift“. Wenn die Bibel als Wort Gottes bezeichnet wird, heißt das, dass sie von Gott kam und nicht von Menschen.
Sie kam wohl über menschliche Kanäle zu uns, doch ist sie ein Wort von Gott. Der Begriff „die Schrift“ ist etwas einengend, weil es sich nur auf das bezieht, was schriftlich fixiert wurde. 
Gott hat vieles gesagt, was nicht schriftlich fixiert wurde. 
Doch in Folge eines göttlichen Entscheids enthält die Bibel die Dinge, die er sagte und die er in schriftlicher Form festhalten lassen wollte. 
Das ist die Schrift: Das was niedergeschrieben ist.

Und über diese Schrift macht Jesus eine einfache, radikale Aussage: Die Schrift kann nicht gebrochen werden. 
Man kann über die göttliche Inspiration oder die Autorität der Bibel debattieren so lange man will, doch Jesus hat bereits alles gesagt: Sie kann nicht gebrochen werden, sie hat absolute Autorität; sie wird restlos erfüllt werden. 
Alles was in ihr steht wird sich exakt bewahrheiten. Du kannst dich gegen sie aussprechen, du kannst sie abstreiten, aber du kannst die Schrift nicht brechen. 
Ja, wenn du sie leugnest, wird sie letztendlich dich brechen. Die Schrift kann nicht gebrochen werden.

Ich möchte euch bitten, einmal gemeinsam mit mir diesen Satz zu sagen: „Die Schrift kann nicht gebrochen werden!“ 
Und nun schau deinem Nachbarn in die Augen und sag nochmal zu ihm: „Die Schrift kann nicht gebrochen werden!“ 
Gut, damit ist der Fall erledigt. 
Wisst ihr, es gibt diese sogenannte Bibelkritik, die aberwitzige Fantasien rund um die Schrift aufbaut, und aus ihr letztlich ein total wirkungsloses, belangloses Buch macht. 
Dem Teufel ist viel daran gelegen, in deinem und meinem Leben unseren Glauben an die Autorität und Exaktheit der Bibel zu unterminieren. 
Doch wenn wir dem Vorbild Jesu folgen, sagen wir einfach nur: „Die Schrift kann nicht gebrochen werden!“ 
Hörst du mich, Satan? 
„Die Schrift kann nicht gebrochen werden!“

\section{Jesus ist Gottes Wort in Person}

Nun, denn. Als nächstes möchte ich euch vor Augen führen, dass nicht nur die Bibel das Wort Gottes ist, sondern auch Jesus selbst ist das Wort Gottes. 
Das wird aus zwei Passagen im Johannesevangelium deutlich. 
Diese bekannten Worte aus Johannes 1:

\begin{quotation}
  Im Anfang war das Wort und das Wort war bei Gott und das Wort war Gott.
  \sourceatright{Joh 1,1}
\end{quotation}

Damit ist Jesus gemeint. 
Er war das Wort, er ist das Wort. 
Und später heißt es:

\begin{quotation}
  Und das Wort ward Fleisch und wohnte unter uns, und wir haben seine Herrlichkeit angeschaut.
  Eine Herrlichkeit als eines eingeborenen vom Vater, voller Gnade und Wahrheit.
  \sourceatright{Joh 1,14}
\end{quotation}

Als Jesus geboren wurde, jene Zeit an die wir zu Weihnachten gedenken, obwohl es nicht diese Zeit wahr, wurde das Wort Fleisch. 
Aber er war immer schon das Wort, von Ewigkeit her war er das Wort, das bei Gott war. 
Und wenn er zurückkommt, wisst ihr, wie er zurückkomme? 
Wisst ihr welchen Namen er tragen wird? 
Ich sag‘s euch: Offenbarung 19. 
Es ist schon bemerkenswert, als er das erste Mal kam, war er das Wort, und wenn er wiederkommt wird er auch das Wort sein. 
Dieser Vers zeichnet ein Bild von Jesus der in Herrlichkeit vom Himmel herabkommt, um sein Reich auf Erden zu errichten. 

\begin{quotation}
  Und ich sah den Himmel geöffnet, und siehe, ein weißes Pferd, und der darauf saß, heißt ‚Der Treue und der Wahrhaftige‘; und in Gerechtigkeit richtet und kämpft er. Seine Augen aber sind wie eine Feuerflamme, und auf seinem Haupt sind viele Kronen, und er trägt einen Namen geschrieben, den niemand kennt als nur er selbst. 
  Und er ist bekleidet mit einem Gewand, das in Blut getaucht ist, und sein Name heißt: ‚Das Wort Gottes‘.
  \sourceatright{Offenbarung 19,11-13}
\end{quotation}

Er war immer das Wort, und ist noch immer das Wort, wird immer das Wort sein. Somit kommen wir zu einer sehr wichtigen Schlussfolgerung: zwischen Jesus und der Bibel herrscht völlige Übereinstimmung. 
Wie du zu Jesus stehst, so stehst du auch zur Bibel. 
Du kannst nicht einerseits an Jesus und andererseits nicht an die Bibel glauben. 
Kannst du dies nachvollziehen? Jesus ist das Wort Gottes. 
Er ist das Fleischgewordene Wort. 
Die Bibel ist das niedergeschriebene, das schriftlich fixierte Wort. 
Du musst zu dem einen genauso stehen wie zu dem anderen. 
Es herrscht absolute Übereinstimmung zwischen den beiden.

Nun, da wir uns dem Ende nähern, möchte ich anhand von Johannes 14 fünf wesentliche Fakten über das Wort Gottes und deine Beziehung zu ihm veranschaulichen. 
In den Versen 19 bis 22 nimmt Jesus in gewisser Hinsicht Abschied von seinen Jüngern. 
Er kündigt ihnen an, dass er bald von ihnen weg gehen würde und sie eine Zeit lang sich selbst überlassen sein würden. 
Dieser Moment hat etwas Traumatisches für die Jünger. 
Sie können diese Offenbarung Jesu nicht verarbeiten. 
Doch während Jesus ihnen dies unterbreitet enthüllt er auf wunderbare Weise welche Bedeutung die Bibel für uns Christen haben sollte. 
Er sagt:

\begin{quotation}
  Noch eine kleine Weile, und die Welt sieht mich nicht mehr; ihr aber seht mich; weil ich lebe, sollt auch ihr leben!
  \sourceatright{Joh 14,19}
\end{quotation}

Hier unterscheidet Jesus zwischen der Welt, allen Menschen, die Jesus nicht anerkennen, und seinen Jüngern. 
Er sagt, die Welt würde ihn nicht sehen, seine Jünger jedoch schon. 
Dann stellt ihm Judas, nicht der Ischariot sondern der andere, in Vers 22 eine sehr entscheidende Frage:

\begin{quotation}
  Da spricht Judas – nicht der Ischariot – zu ihm: Herr, wie kommt es, dass du dich uns offenbaren willst und nicht der Welt?
  \sourceatright{Joh 14,22}
\end{quotation}

Damit greift er die Aussage Jesu auf, die Welt würde ihn nicht sehen, die Jünger jedoch schon. Seine Frage war, „wie wirst du dich uns offenbaren der Welt jedoch nicht“? 
Die Antwort Jesu steckt voller wichtiger Wahrheiten. Jesus antwortete und sprach:

\begin{quotation}
  Wenn jemand mich liebt, so wird er mein Wort befolgen, und mein Vater wird ihn lieben, und wir werden zu ihm kommen und Wohnung bei ihm machen.
  \sourceatright{Joh 14,23}
\end{quotation}

\section{Das Wort Gottes halten}

Ich möchte aus dieser Antwort Jesu fünf absolut entscheidende Aspekte heraus greifen:

\subsection{Gott offenbart sixh}

Erstens, zunächst einmal sagte Jesus er würde sich seinen Jüngern offenbaren und nicht der Welt. 
Wodurch unterscheiden sich die Jünger von der Welt? 
Die Antwort ist: Dadurch, dass sie das Wort Gottes halten. 
Wahre Jünger halten das Wort Jesu. 
Ihr Kennzeichen ist nicht ihre Konfession, sondern ihre Haltung gegenüber dem Wort. 
Dieses Kriterium entscheidet darüber, ob du ein wahrer Jünger bist oder nicht: Deine Beziehung zum Wort Gottes.

Jünger unterscheiden sich dadurch von der Welt, dass sie das Wort Gottes halten. 
Jeder einzelne von uns ist einer dieser beiden Kategorien zu zurechnen: Wenn wir Jünger sind, halten wir das Wort Gottes; wenn wir das Wort Gottes nicht halten, gehören wir zur Welt – zu der Welt, die der Herrschaft Jesu nicht untersteht.

\subsection{Die Liebe zu Gott – Beweggrund des Gehorsams}

Die zweite Wahrheit formuliert Jesus so: „Wenn jemand mich liebt, so wird er mein Wort halten.“ 
Ob ein Jünger das Wort Gottes hält oder nicht ist demnach der Prüfstein seiner Liebe zu Gott. Die Liebe ist der Beweggrund für den Gehorsam. 
Es ist sehr wichtig zu verstehen, dass wir Christen uns nicht von Furcht leiten lassen, sondern von Liebe. Das Gesetz bediente sich in gewisser Hinsicht der Motivation Furcht: Wenn du das und das tust, wirst du bestraft. 
Doch das funktioniert nicht. 
Ich habe mitgeholfen eine ganze Menge Kinder groß zu ziehen. 
Ich habe dabei festgestellt, dass wir Eltern bei der Erziehung mit Furcht arbeiten können, solange die Kinder unserer Obhut unterstehen. 
Doch sobald sie uns verlassen haben, werden sie einen ganz anderen Kurs einschlagen, wenn ihr Leben zuvor von Furcht bestimmt war. 
Die einzige Motivation, die sie uns gegenüber treu und loyal sein lässt, ist die Liebe. 
Gott und Jesus waren weise genug, nicht auf Furcht sondern auf Liebe zu bauen. 
Somit erweist sich die Liebe des Jüngers zu Gott vor allem darin, dass er dessen Wort hält. 
„Wenn jemand mich liebt, so wird er mein Wort halten.“ Die Liebe ist der Beweggrund für den Gehorsam.

\subsection{Geliebt vom Vater}

Drittens sagt Jesus, „Wenn jemand mich liebt wird er mein Wort halten, und mein Vater wird ihn lieben“.
Das ist eine weitere herrliche Wahrheit. Wenn wir das Wort Gottes halten, wird Gott Vater uns mit einer ganz besonderen Liebe lieben.
Gott liebt zwar in gewisser Hinsicht ohnehin die ganze Welt, doch für die wahren Jünger Jesu, für jene, die sein Wort halten, hat Gott eine ganz andere und intensivere Liebe.

\subsection{Offenbarung durch das Wort}

Viertens haben wir da noch die Frage des Judas: „Herr, wie kommt es, dass du dich uns offenbaren willst und nicht der Welt?“
Worauf Jesus antwortet: „Wenn jemand mich liebt wird er mein Wort halten.“ Wie offenbart sich Christus demnach uns?
Durch das Wort. 
Indem wir das Wort halten, lernen wir Jesus besser kennen. 
Vielleicht machen wir eine herrliche, geistliche Erfahrung, vielleicht werden wir in den dritten Himmel entrückt oder etwas ähnliches. 
Doch die meisten Leute erleben so etwas nicht, und das ist auch nicht die vorrangige Art und Weise, wie Gott und Jesus sich uns offenbaren. 
Vielmehr offenbaren sie sich uns, indem wir das Wort Gottes halten.

\subsection{Gott nimmt Wohnung bei (in) uns auf}

Fünftens haben wir zum Abschluss noch folgende, erstaunliche Aussage: „Wenn jemand mich liebt, so wird er mein Wort befolgen, und mein Vater wird ihn lieben, und wir werden zu ihm kommen und Wohnung bei ihm machen.“ 
Nur an sehr wenigen Stellen in der Bibel steht, dass Pronomen für Gott im Plural; das ist so eine Stelle. 
Jesus sagt: „Wir, mein Vater und ich, werden zu ihm kommen und Wohnung bei ihm machen“. 
Es ist eine atemberaubende Aussage, eine erstaunliche Offenbarung, dass Gott Vater und Gott Sohn zu uns kommen, und bei uns Wohnung machen wollen. 
Sie wollen uns zu ihrer persönlichen Wohnung machen. 
Wie kommt dies zustande? 
Durch das Wort. 
Wenn wir das Wort nicht lieben, wenn wir dem Wort nicht gehorchen, wird Gott seine Wohnung nicht bei uns machen.

\section{Zusammenfassung}

Ich möchte diesen Vortrag mit diesem sehr ernsten Gedanken abschließen: Du liebst Gott genau so viel, wie du sein Wort liebst. 
Wenn du nun wissen möchtest, wie sehr du eigentlich Gott liebst, und welchen Platz er in deinem Leben hat, dann kannst du das sehr einfach ausfindig machen. Du brauchst darüber nicht zu spekulieren; frag dich einfach, „wie sehr liebe ich die Bibel?
Welchen Stellenwert hat die Bibel in meinem Leben?“
Genau so sehr liebst du Gott, und genau diesen Stellenwert räumst du Gott in deinem Leben ein.

Ich möchte diese fünf Aussagen über die Bibel noch einmal zusammenfassen, weil sie von entscheidender Bedeutung sind. 
Wisst ihr, viele Christen leben in einer Grauzone. Sie wissen im Grunde nicht was Licht und was Finsternis ist. 
Sie wünschen es sich, sie hoffen es, aber sie sind sich nicht sicher. 
Das liegt daran, dass sie in ihrem Leben, dem Wort Gottes nicht seinen rechtmäßigen Platz eingeräumt haben.

Nun noch einmal diese fünf Aussagen, und dann schließen wir:

\begin{enumerate}
  \item Wahre Jünger unterscheiden sich dadurch von der Welt, dass sie das Wort Gottes halten.
  \item Die Liebe eines Jüngers zu Gott erweist sich vor allem darin, dass er das Wort Gottes hält.
  Nicht Furcht sondern Liebe ist der Beweggrund für unseren Gehorsam.
  \item Gott liebt den Jünger vor allem deshalb, weil dieser sein Wort hält.
  Gott hat eine spezielle Liebe zu Jüngern. 
  Er liebt die ganze Welt, hat jedoch eine spezielle Liebe für einen Jünger.
  Doch jene die er als seine Jünger liebt sind jene, die Gottes Wort halten
  Wenn du einen ganz besonderen Platz bei Gott haben möchtest, dann musst du sein Wort halten.
  \item Indem wir das Wort Gottes halten und ihm gehorchen, offenbart sich uns Christus. 
  Die Frage lautete: „Herr, wie kommt es, dass du dich uns offenbaren willst und nicht der Welt?“ 
  Jesus sagte: „Wenn jemand mich liebt wird er mein Wort halten, und so werde ich mich offenbaren.“
  \item Gott Vater und Gott Sohn werden durch das Wort Gottes bei uns Wohnung machen.
  Ist das nicht ein erstaunlicher Gedanke? 
  Mir verschlägt es dabei die Sprache! 
  Gott Vater und Gott Sohn wollen bei uns Wohnung machen, doch sie tun das nur, wenn wir das Wort Gottes halten.
\end{enumerate}

Zum Abschluss möchte ich für uns alle beten:

\begin{quotation}
  Himmlischer Vater, wir danken dir für dein Wort, das Wort Gottes, die Bibel, für dieses sichere, unfehlbare Wort, mit uneingeschränkter Autorität, das eine Leuchte für unseren Fuß und ein Licht auf unserem Weg ist.
  Ich möchte für jeden beten, der heute hier ist und für jeden der auf irgendeine Weise mit dieser Botschaft konfrontiert wird: Mach uns zu Menschen, die dein Wort lieben.
  Versetze uns durch deine Liebe in die Lage, deinem Wort, der Bibel, seinen rechtmäßigen Platz in unserem Leben zu geben, damit wir echte Jünger des Herrn Jesus sind.
  Im Namen Jesu, Amen.
\end{quotation}
